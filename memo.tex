%%%%%%%%%%%%%%%%%%%%%%%%%%%%%%%%%%%%%%%%%%%%%%%%%%%%%%%%%%%%%%%%%%%%%%%%%%%%%%%%
%2345678901234567890123456789012345678901234567890123456789012345678901234567890
%        1         2         3         4         5         6         7         8
%% memo.tex
%% V1.0
%% 2015/09/20
%% by Prof. Rui Santos Cruz
%% based on a template created by Rob Oakes
%%%%%%%%%%%%%%%%%%%%%%%%%%%%%%%%%%%%%%%%%%%%%%%%%%%%%%%%%%%%%%%%%%%%%%%%%%%%%%%%
\documentclass[a4paper,12pt]{texMemo}
% --> Please Choose the MAIN LANGUAGE for the document in package BABEL
% --> by replacing "main=" in language name selector. Default is "main=english"
\usepackage[main=english,portuguese]{babel} % Defines Main language
\usepackage[utf8]{inputenc}
\usepackage{iflang}
\usepackage{ifthen}
\usepackage{parskip}
\setlength{\parindent}{15pt}
\logo{\includegraphics[width=0.3\textwidth]{madison.jpg}} % Logo
%%%%%%%%%%%%%%%%%%%%%%%%%%%%%%%%%%%%%%%%%%%%%%%%%%%%%%%%%%%%%%%%%%%%%%%%%%%%%%%%
%	MEMO INFORMATION --> Write your info in the following tags.
%%%%%%%%%%%%%%%%%%%%%%%%%%%%%%%%%%%%%%%%%%%%%%%%%%%%%%%%%%%%%%%%%%%%%%%%%%%%%%%%
\memofrom{Luxi Cao, Zongyan Wang} % Sender(s) Name
\memocourse{Stat 992: Dependence Modeling with Copulas} % Course Name, or abbreviated acronym
\memosubject{Lecture 7 Scribe} % Subject
\memodate{\today} % Date, -> set to \today for automatically print todays date
%%%%%%%%%%%%%%%%%%%%%%%%%%%%%%%%%%%%%%%%%%%%%%%%%%%%%%%%%%%%%%%%%%%%%%%%%%%%%%%%
\begin{document}
\maketitle % Print the memo header information
%%%%%%%%%%%%%%%%%%%%%%%%%%%%%%%%%%%%%%%%%%%%%%%%%%%%%%%%%%%%%%%%%%%%%%%%%%%%%%%%
%	MEMO CONTENT --> Your content is written here
%%%%%%%%%%%%%%%%%%%%%%%%%%%%%%%%%%%%%%%%%%%%%%%%%%%%%%%%%%%%%%%%%%%%%%%%%%%%%%%%
Q1.

\begin{enumerate}
\item As long as $F^q, q>0$, $F$ is infinitely divisible
\item Domain of Attraction: Limit distribution of Minimum or Maximum of random variables exists
\end{enumerate}

Q2.1. How to relate infinitely divisible, domain of attraction, stable, slowly varying?
1)In probability theory, a probability distribution is infinitely divisible if it can be expressed as the probability distribution of the sum of an arbitrary number of independent and identically distributed random variables. 
2)
multivariate distribution function,
then it can be considered as infinitely divisible.
domain of attraction: 
stable:
sum stable
IF a sequesce of rv has indivisible proporty
unit of frechet dist
slowly varying: a function that 
many of those joint probability, can be expressed as survival probability.

2. 
If our estimation procedure is consistent then both good enough.
Some people dislike 2 stage, because automatically it isn't good.
joint likelihood, we do 2 stage.
coordinate decent is more popular recently.
answer varies for "big".

3. chap3 would talk about it

4. AC property will be shown. broad family, it generates gumble C which is extreme ones.

5. CO usually what we know is joint dist, 
copula dependence b/c in general we don't have uni within the same family
multivariate joint dist->marginal distribution
we can transform any marginal we want. 
we fit some dist to marginal and transform it to frechet. 
application will be shown.

6. 7. going to talk about it.

text p28: 1-F=delta1Fac+delta2Fsin+delta3Fd those three delta are different

8.VC will later be talked about
estimate PC, hold off


9. apply C, eg:
$Y=(y_1,y_2,....,y_n)$    vector $\lambda(x_1,x_2,x_3,..,x_p)=f(a_1x_1+a_2x_2+...a_px_p)$
multivariavte normal choose lambda as mu would be the same thing.

10. compare C and Multivariate dist fn
MD can also be considered as C, but it doesn't has uniform dist
After we have transformed to the same distribution, 

11. l(tx)/l(x) has a limit 1. eg. logx, 

12. 3 distinct regions:
In general linear regression, many situations we always have some good model fitting. 
Ratio estimation procedure, good estimation at upp tail.
No concrete answer.
Extreme dependence no much attention to the middle part.
Regression pay attention to the middle part, which dominates the model. 
We don't use residual, we check misclassification rate. 
If you generate data using the hyperplane,. either solution is acceptable for classification. 

13. VC in 2.6 chp4 talk more

14. high order moment
C trandform MV to uniform, the high order moment exists.

15. CC in a subset we choose C. Up to our purpose.

16. truncation 
orig: model fitting, truncation, 
T,C applies to univariate marginals 
tail dependence: $P(X>u|Y>u)$
Bivariate normal,  $lim_{u->u^*}P(X^*>u|Y^*>u)=\lambda?0$
$X^*=min{X,u^*},Y^*=min{Y,u^*}$   

17. Effect of misspecification
Truncation and censoring effect: Research question more like survival analysis part.
Basically, truncation adn censoring are different. If we treat truncated data as normal distribution and transform it to uniform distribution, then we don't know what will happen yet. It also depends on the types of model we have. It's good research question. 
18.frechet class:
eg:copula fn 
first one. 
Upper bound:always is a joint distribution, lower bound: not always a joint distribution

Q3.
%%%%%%%%%%%%%%%%%%%%%%%%%%%%%%%%%%%%%%%%%%%%%%%%%%%%%%%%%%%%%%%%%%%%%%%%%%%%%%%%
\end{document}